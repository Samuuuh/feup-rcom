%----------------------------------------------------------------------------------------
%	PACKAGES AND DOCUMENT CONFIGURATIONS
%----------------------------------------------------------------------------------------
\documentclass[article, a4paper, 11pt, oneside]{memoir}

% Margins
\usepackage[top=3cm,left=2cm,right=2cm,bottom=3cm]{geometry}

% Encondings
\usepackage[utf8]{inputenc}

% Language
\usepackage[portuguese]{babel}

% Graphics and images
\usepackage{graphicx}
	\graphicspath{{./images/}}

% Listings
\usepackage{listings}
\usepackage{amsmath}

% Color
\usepackage[dvipsnames]{xcolor}

% Tables
\usepackage{tabularx}

% Math symbols
\usepackage{amssymb}

% Paragraph Spacing
\usepackage{parskip}
\usepackage{indentfirst}
\setlength{\parskip}{0.5cm}

% Hyperreferences
\usepackage{hyperref}

% Repeated commands
\usepackage{expl3}
\ExplSyntaxOn
\cs_new_eq:NN \Repeat \prg_replicate:nn
\ExplSyntaxOff

% Header and Footer Things
\usepackage{wallpaper}
\usepackage{fancyhdr}

% Following code to edit the pagestyle
\pagestyle{fancy}
\fancyhf{}
\rhead{RCOM}
\lhead{\leftmark}
\rfoot{Página \thepage}

% Commands
\usepackage{xargs}

%% Linked Email
\newcommand{\email}[1]{
{\texttt{\href{mailto:#1}{#1}} }
}

%----------------------------------------------------------------------------------------
%	DOCUMENT INFORMATION
%----------------------------------------------------------------------------------------
% Title
\title{\Huge \texttt{Rede de Computadores} }
% Authors
\author{
\LARGE \textbf{Turma Grupo}\\\\
\begin{tabular}{l r}
	\email{up201806250@fe.up.pt} & Diogo Samuel Gonçalves Fernandes	\\
	\email{up201806505@fe.up.pt} & Paulo Jorge Salgado Marinho Ribeiro \\
\end{tabular}
}

% Date for the report
\date{\today}

% Table of Contents
\addto\captionsportuguese{\renewcommand*\contentsname{Índice}}

%----------------------------------------------------------------------------------------
%	DOCUMENT
%----------------------------------------------------------------------------------------
\begin{document}
%----------------------------------------------------------------------------------------
%	Front Page
%----------------------------------------------------------------------------------------
% Title Author and Date
\maketitle

% More information for front page
\begin{center}
\textbf{Projeto RCOM - 2019/20 - MIEIC}
\Repeat{2}{\linebreak}
\begin{tabular}{l r}
	\textbf{Professor das Aulas Práticas}: 
\end{tabular}
\Repeat{4}{\linebreak}

\end{center}

\newpage
%----------------------------------------------------------------------------------------
%	CHAPTER - Descrição do Problema
%----------------------------------------------------------------------------------------

Informação sobre o relatório do 1º trabalho laboratorial de RCOM

a) O relatório do 1º trabalho laboratorial de RCOM deverá conter a seguinte informação:

- Título, Autores

- Sumário
  (dois parágrafos: um sobre o contexto do trabalho; outro sobre as principais conclusões do relatório)

\newpage
%----------------------------------------------------------------------------------------
%	CHAPTER 1 - Descrição do Problema
%----------------------------------------------------------------------------------------
\chapter[Introdução][Introdução]{Introdução} \label{\thechapter}

1. Introdução
  (indicação dos objectivos do trabalho e do relatório; descrição da lógica do relatório com indicações sobre o tipo de informação que poderá ser encontrada em cada uma secções seguintes)

\newpage
%----------------------------------------------------------------------------------------
%	CHAPTER 2 - Arquitetura
%----------------------------------------------------------------------------------------
\chapter[Arquitetura][Arquitetura]{Arquitetura} \label{\thechapter}

2. Arquitetura
  (blocos funcionais e interfaces)

\newpage
%----------------------------------------------------------------------------------------
%	CHAPTER 3 - Estrutura do código
%----------------------------------------------------------------------------------------
\chapter[Estrutura do código][Estrutura do código]{Estrutura do código} \label{\thechapter}

3. Estrutura do código
  (APIs, principais estruturas de dados, principais funções e sua relação com a arquitetura)

\newpage
%----------------------------------------------------------------------------------------
%	CHAPTER 4 - Casos de uso principais
%----------------------------------------------------------------------------------------
\chapter[Casos de uso principais][Casos de uso principais]{Casos de uso principais} \label{\thechapter}


4. Casos de uso principais
  (identificação; sequências de chamada de funções)

\newpage
%----------------------------------------------------------------------------------------
%	CHAPTER 5 - Protocolo de ligação lógica
%----------------------------------------------------------------------------------------
\chapter[Protocolo de ligação lógica][Protocolo de ligação lógica]{Protocolo de ligação lógica} \label{\thechapter}

5. Protocolo de ligação lógica
  (identificação dos principais aspectos funcionais; descrição da estratégia de implementação destes aspectos com apresentação de extratos de código)

\newpage
%----------------------------------------------------------------------------------------
%	CHAPTER 6 - Protocolo de aplicação
%----------------------------------------------------------------------------------------
\chapter[Protocolo de aplicação][Protocolo de aplicação]{Protocolo de aplicação} \label{\thechapter}
  
6. Protocolo de aplicação
  (identificação dos principais aspectos funcionais; descrição da estratégia de implementação destes aspectos com apresentação de extractos de código)

\newpage
%----------------------------------------------------------------------------------------
%	CHAPTER 7 - Validação
%----------------------------------------------------------------------------------------
\chapter[Validação][Validação]{Validação} \label{\thechapter}

7. Validação
  (descrição dos testes efectuados com apresentação quantificada dos resultados, se possível)

\newpage
%----------------------------------------------------------------------------------------
%	CHAPTER 8 - Estrutura do código
%----------------------------------------------------------------------------------------
\chapter[Eficiência do protocolo de ligação de dados][Eficiência do protocolo de ligação de dados]{Eficiência do protocolo de ligação de dados} \label{\thechapter}

8. Eficiência do protocolo de ligação de dados (caraterização estatística da  eficiência do protocolo, feita com recurso a medidas sobre o código desenvolvido. 
A caracterização teórica de um protocolo Stop Wait, que 
   deverá ser usada como termo de comparação, encontra-se descrita nos slides de Ligação Lógica das aulas teóricas).

\newpage
%----------------------------------------------------------------------------------------
%	CHAPTER 9 - Conclusões
%----------------------------------------------------------------------------------------
\chapter[Conclusões][Conclusões]{Conclusões} \label{\thechapter}

9. Conclusões
  (síntese da informação apresentada nas secções anteriores; reflexão sobre os objectivos de aprendizagem alcançados)

- Anexo I - Código fonte
- Outros anexos, se necessário


b) O estudante tem a liberdade de "fundir" algumas das secções apresentadas (Sec. 2 a Sec.8). Poderá, por exemplo, despromover algumas das secções a subsecções.


c) O relatório não pode exceder 8 páginas A4, fonte de 11pt. Os anexos não estão incluidos nestas 8 páginas.


d) O relatório pode ser entregue até uma semana depois da demonstração ter sido aceite pelo professor.


e) Os relatórios devem ser submetidos via moodle, na turma correspondente. O grupo de alunos deverá, no entanto, facultar uma cópia do relatório em papel ao professor, caso este lhe seja pedido.


f) Algumas dicas sobre o estilo de escrita a adoptar em relatórios técnicos:


1. a escrita deve ser organizada em torno do parágrafo.
  Cada parágrafo deve ser dedicado a um único assunto e,
  na primeira linha do parágrafo, deve poder perceber-se
  de imediato qual é esse assunto;

2. o parágrafo deve ser grande. Este estilo de escrita permite
  uma leitura "fotográfica" da página;

3. escreva pela positiva; descreva o que fez e não o que não fez;

4. seja cuidadoso com a utilização de adjetivos e evite
  descrições vagas.  Não diga que  o seu código é rápido; diga
  que determinada função é executada em cerca de 200 ms. Não diga
  que fez muito código; diga que fez cerca de 800 linhas de código;

5. depois de escrever uma frase ou um parágrafo releia-o. Se
  conseguir colocar a mesma informação num menor número de
  palavras, faça-o;

6. Não use uma segunda frase para explicar de outra forma o assunto
  que descreveu na primeira frase; nesta situação deverá re-escrever
  a primeira frase;

7. Se, durante a escrita, sentir necessidade de abrir parenteses ou
  introduzir um parágrafo adicional para introduzir um qualquer
  conceito, reformule o texto. Provavelmente estes conceitos deveriam
  ter sido introduzidos numa secção ou num capítulo anterior. As frases e
  os parágrafos devem aparecer sempre em sequência natural;

8. Escreva o texto como se estivesse a fazer software. Defina nomes
    e conceitos antes de os usar e depois, no texto, use sempre o mesmo nome
  para o mesmo conceito, mesmo que tenha de o usar múltiplas vezes na
  mesma frase.
  


\end{document}