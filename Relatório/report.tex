%----------------------------------------------------------------------------------------
%	PACKAGES AND DOCUMENT CONFIGURATIONS
%----------------------------------------------------------------------------------------
\documentclass[article, a4paper, 11pt, oneside]{memoir}

% Margins
\usepackage[top=3cm,left=2cm,right=2cm,bottom=3cm]{geometry}

% Encondings
\usepackage[utf8]{inputenc}

% Language
\usepackage[portuguese]{babel}

% Graphics and images
\usepackage{graphicx}
	\graphicspath{{./images/}}

% Listings
\usepackage{listings}
\usepackage{amsmath}

% Color
\usepackage[dvipsnames]{xcolor}

% Tables
\usepackage{tabularx}

% Math symbols
\usepackage{amssymb}

% Paragraph Spacing
\usepackage{parskip}
\usepackage{indentfirst}
\setlength{\parskip}{0.5cm}

% Hyperreferences
\usepackage{hyperref}

% Repeated commands
\usepackage{expl3}
\ExplSyntaxOn
\cs_new_eq:NN \Repeat \prg_replicate:nn
\ExplSyntaxOff

% Header and Footer Things
\usepackage{wallpaper}
\usepackage{fancyhdr}

% Following code to edit the pagestyle
\pagestyle{fancy}
\fancyhf{}
\rhead{RCOM}
\lhead{\leftmark}
\rfoot{Página \thepage}

% Commands
\usepackage{xargs}

%% Linked Email
\newcommand{\email}[1]{
{\texttt{\href{mailto:#1}{#1}} }
}

%----------------------------------------------------------------------------------------
%	DOCUMENT INFORMATION
%----------------------------------------------------------------------------------------
% Title
\title{\Huge \texttt{Rede de Computadores} }
% Authors
\author{
\LARGE \textbf{Turma Grupo}\\\\
\begin{tabular}{l r}
	\email{up201806250@fe.up.pt} & Diogo Samuel Gonçalves Fernandes	\\
	\email{up201806505@fe.up.pt} & Paulo Jorge Salgado Marinho Ribeiro \\
\end{tabular}
}

% Date for the report
\date{\today}

% Table of Contents
\addto\captionsportuguese{\renewcommand*\contentsname{Índice}}

%----------------------------------------------------------------------------------------
%	DOCUMENT
%----------------------------------------------------------------------------------------
\begin{document}
%----------------------------------------------------------------------------------------
%	Front Page
%----------------------------------------------------------------------------------------
% Title Author and Date
\maketitle

% More information for front page
\begin{center}
\textbf{Projeto RCOM - 2019/20 - MIEIC}
\Repeat{2}{\linebreak}
\begin{tabular}{l r}
	\textbf{Professor das Aulas Práticas}: 
\end{tabular}
\Repeat{4}{\linebreak}

\end{center}

\newpage
%----------------------------------------------------------------------------------------
%	CHAPTER - Descrição do Problema
%----------------------------------------------------------------------------------------

Informação sobre o relatório do 1º trabalho laboratorial de RCOM

a) O relatório do 1º trabalho laboratorial de RCOM deverá conter a seguinte informação:

- Título, Autores

- Sumário
  (dois parágrafos: um sobre o contexto do trabalho; outro sobre as principais conclusões do relatório)

\newpage
%----------------------------------------------------------------------------------------
%	CHAPTER 1 - Descrição do Problema
%----------------------------------------------------------------------------------------
\chapter[Introdução][Introdução]{Introdução} \label{\thechapter}

Tendo como principal objetivo implementar um protocolo de transferência de dados recorrendo a uma porta série, este trabalho deve resultar num programa capaz de resistir a fenómenos como a interrupção da porta série ou a receção de informação corrompida, provocada pela indução de “ruído” na porta série. Este relatório procura explicar toda a teoria envolvida neste primeiro trabalho, de forma bem estrutura, nos seguintes tópicos:
 - Arquitetura
	Descrição dos blocos funcionais e interfaces
	- Estrutura do código
	Explicação das APIs, enumeração das principais estruturas de dados utilizadas, das funções de maior importância e relação com a arquitetura
	- Casos de Uso Principais
	Identificação dos casos de uso mais importantes, e demonstração sequencial das chamadas às funções.
	- Protocolo de ligação lógica
	Identificação dos principais aspetos funcionais da camada de Ligação de Dados, descrição da estratégia de implementação destes aspetos, com o apoio de extratos de código.
	- Protocolo de aplicação
	Identificação dos principais aspetos funcionais da camada da Aplicação, descrição da estratégia de implementação destes aspetos, com o apoio de extratos de código.
- Validação
Descrição dos testes efetuados com apresentação quantificada dos resultados.
- Eficiência do protocolo de ligação de dados
Caracterização estatística da eficiência do protocolo, feita com recurso a medidas sobre o código desenvolvido.
- Conclusões
Síntese da informação apresentada nas secções anteriores, e reflexão sobre os objetivos de aprendizagem alcançados.

\newpage
%----------------------------------------------------------------------------------------
%	CHAPTER 2 - Arquitetura
%----------------------------------------------------------------------------------------
\chapter[Arquitetura][Arquitetura]{Arquitetura} \label{\thechapter}

O programa desenvolvido desdobra-se em duas camadas bem definidas: a camada de Ligação de Dados (Link Layer) e a camada da Aplicação (Application Layer). 
A primeira, como é responsável pelo estabelecimento da ligação, torna o protocolo sólido, garantindo a sua consistência. Por este motivo, é considerada a camada de mais baixo nível do programa, sendo que trata da abertura da porta série, da transmissão de informação (escrita e leitura), e do seu posterior fecho. É também da sua responsabilidade testar se a informação foi escrita/recebida corretamente, através de byte stuffing, e de testes de erros como os BCC, que serão aqui detalhados mais tarde.Por outro lado, a camada da Aplicação é apenas responsável pelo envio e receção da informação dos ficheiros, pelo que é de um nível superior à camada de ligação de dados. Assim, esta camada chama as funções da camada da ligação de dados, para envio/receção da informação de dados, mantendo-se, no entanto, completamente independente desta, uma vez que desconhece os seus métodos de atuar.

\newpage
%----------------------------------------------------------------------------------------
%	CHAPTER 3 - Estrutura do código
%----------------------------------------------------------------------------------------
\chapter[Estrutura do código][Estrutura do código]{Estrutura do código} \label{\thechapter}

O Makefile por nós criado efetua a compilação do programa, resultando em dois executáveis diferentes, um para o Emissor (writenoncanonical) e outro para o Recetor (noncanonical).
Emissor:
O executável relativo ao Emissor exige 2 argumentos: o nome da porta série (por exemplo /dev/ttyS1), e o nome do ficheiro que vai ser transmitido (exemplo: pinguim.gif)
A sequência de chamadas efetuada por este executável é a seguinte:
 - llopen: Configura a ligação entre os dois computadores, abrindo a porta série em modo de escrita e leitura. Esta configuração decorre de uma troca de tramas, a trama SET enviada pelo Emissor e a trama UA enviada pelo Recetor. Apesar de a função ser comum aos dois programas, há nela uma distinção das ações conforme o parâmetro status da struct referida no tópico anterior, recebida como parâmetro.
 - Leitura do ficheiro e armazenamento da sua informação num array de unsigned chars.
 - Criação do pacote de controlo Start, seguido do seu envio, já recorrendo à função llwrite().
 - Criação dos pacotes de Informação, que resulta de uma divisão do array referido no segundo passo, e o seu respetivo envio, recorrendo também a llwrite().
 - Criação do pacote de controlo End, seguido do seu envio, recorrendo à função llwrite().
 - llclose(): Encerramento da ligação entre os dois computadores, através de uma troca de tramas. Neste caso, o Emissor receberá uma trama DISC, enviando como resposta outra trama DISC, e para terminar receberá uma trama UA.

 Recetor:
 O executavel relativo ao Recetor exige tambem 2 argumentos: o nome da porta serie (por exemplo /dev/ttyS0), e o nome do ficheiro que vai ser transmitido (exemplo pinguim.gif)
 - llopen: Configura a ligação entre os dois computadores, abrindo a porta série em modo de escrita e leitura. Esta configuração decorre de uma troca de tramas, a trama SET enviada pelo Emissor e a trama UA enviada pelo Recetor. Apesar de a função ser comum aos dois programas, há nela uma distinção das ações conforme o parâmetro status da struct referida no tópico anterior, recebida como parâmetro.
- Receção do pacote de controlo Start, recorrendo à função llread().
- Processamento do pacote de controlo Start recebido, de modo a receber corretamente o nome e o tamanho do ficheiro que vai ser copiado, para efeitos de apenas informar o utilizador destes dados.
 - Receção dos pacotes de Informação, recorrendo à função llread(). A cada pacote lido, a sua informação é processada (de modo a ficar apenas com os bytes de informação do ficheiro), e esta informação é logo de seguida escrita para o novo ficheiro, criado imediatamente antes desta receção.
 - Quando recebe o pacote de controlo End, o loop de receção de tramas termina.
 - llclose(): Encerramento da ligação entre os dois computadores, através de uma troca de tramas. Neste caso, o Recetor enviará uma trama DISC, recebendo como resposta outra trama DISC, e para terminar enviará uma trama UA.

\newpage
%----------------------------------------------------------------------------------------
%	CHAPTER 4 - Casos de uso principais
%----------------------------------------------------------------------------------------
\chapter[Casos de uso principais][Casos de uso principais]{Casos de uso principais} \label{\thechapter}


4. Casos de uso principais
  (identificação; sequências de chamada de funções)

\newpage
%----------------------------------------------------------------------------------------
%	CHAPTER 5 - Protocolo de ligação lógica
%----------------------------------------------------------------------------------------
\chapter[Protocolo de ligação lógica][Protocolo de ligação lógica]{Protocolo de ligação lógica} \label{\thechapter}

5. Protocolo de ligação lógica
  (identificação dos principais aspectos funcionais; descrição da estratégia de implementação destes aspectos com apresentação de extratos de código)

\newpage
%----------------------------------------------------------------------------------------
%	CHAPTER 6 - Protocolo de aplicação
%----------------------------------------------------------------------------------------
\chapter[Protocolo de aplicação][Protocolo de aplicação]{Protocolo de aplicação} \label{\thechapter}
  
6. Protocolo de aplicação
  (identificação dos principais aspectos funcionais; descrição da estratégia de implementação destes aspectos com apresentação de extractos de código)

\newpage
%----------------------------------------------------------------------------------------
%	CHAPTER 7 - Validação
%----------------------------------------------------------------------------------------
\chapter[Validação][Validação]{Validação} \label{\thechapter}

7. Validação
  (descrição dos testes efectuados com apresentação quantificada dos resultados, se possível)

\newpage
%----------------------------------------------------------------------------------------
%	CHAPTER 8 - Estrutura do código
%----------------------------------------------------------------------------------------
\chapter[Eficiência do protocolo de ligação de dados][Eficiência do protocolo de ligação de dados]{Eficiência do protocolo de ligação de dados} \label{\thechapter}

8. Eficiência do protocolo de ligação de dados (caraterização estatística da  eficiência do protocolo, feita com recurso a medidas sobre o código desenvolvido. 
A caracterização teórica de um protocolo Stop Wait, que 
   deverá ser usada como termo de comparação, encontra-se descrita nos slides de Ligação Lógica das aulas teóricas).

\newpage
%----------------------------------------------------------------------------------------
%	CHAPTER 9 - Conclusões
%----------------------------------------------------------------------------------------
\chapter[Conclusões][Conclusões]{Conclusões} \label{\thechapter}

9. Conclusões
  (síntese da informação apresentada nas secções anteriores; reflexão sobre os objectivos de aprendizagem alcançados)

- Anexo I - Código fonte
- Outros anexos, se necessário


b) O estudante tem a liberdade de "fundir" algumas das secções apresentadas (Sec. 2 a Sec.8). Poderá, por exemplo, despromover algumas das secções a subsecções.


c) O relatório não pode exceder 8 páginas A4, fonte de 11pt. Os anexos não estão incluidos nestas 8 páginas.


d) O relatório pode ser entregue até uma semana depois da demonstração ter sido aceite pelo professor.


e) Os relatórios devem ser submetidos via moodle, na turma correspondente. O grupo de alunos deverá, no entanto, facultar uma cópia do relatório em papel ao professor, caso este lhe seja pedido.


f) Algumas dicas sobre o estilo de escrita a adoptar em relatórios técnicos:


1. a escrita deve ser organizada em torno do parágrafo.
  Cada parágrafo deve ser dedicado a um único assunto e,
  na primeira linha do parágrafo, deve poder perceber-se
  de imediato qual é esse assunto;

2. o parágrafo deve ser grande. Este estilo de escrita permite
  uma leitura "fotográfica" da página;

3. escreva pela positiva; descreva o que fez e não o que não fez;

4. seja cuidadoso com a utilização de adjetivos e evite
  descrições vagas.  Não diga que  o seu código é rápido; diga
  que determinada função é executada em cerca de 200 ms. Não diga
  que fez muito código; diga que fez cerca de 800 linhas de código;

5. depois de escrever uma frase ou um parágrafo releia-o. Se
  conseguir colocar a mesma informação num menor número de
  palavras, faça-o;

6. Não use uma segunda frase para explicar de outra forma o assunto
  que descreveu na primeira frase; nesta situação deverá re-escrever
  a primeira frase;

7. Se, durante a escrita, sentir necessidade de abrir parenteses ou
  introduzir um parágrafo adicional para introduzir um qualquer
  conceito, reformule o texto. Provavelmente estes conceitos deveriam
  ter sido introduzidos numa secção ou num capítulo anterior. As frases e
  os parágrafos devem aparecer sempre em sequência natural;

8. Escreva o texto como se estivesse a fazer software. Defina nomes
    e conceitos antes de os usar e depois, no texto, use sempre o mesmo nome
  para o mesmo conceito, mesmo que tenha de o usar múltiplas vezes na
  mesma frase.
  


\end{document}